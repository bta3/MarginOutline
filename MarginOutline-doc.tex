% MarginOutline documentation
% created by Byron Ahn
% email address: switch the words before and after the period in ahn.byron@gmail.com
% 2018.02.07
\documentclass[11pt]{article}
\usepackage[varvertical,constantsize,allbullets]{MarginOutline}
\usepackage[margin=1.25in]{geometry}
\usepackage[bottom]{footmisc}
%\usepackage{fancyvrb}
%	\fvset{%
%	fontsize=\footnotesize,
%	numbers=left
%	}
\usepackage{minted}
	\fvset{%
	fontsize=\footnotesize,
	numbers=left,
	breaklines=true
	}
\usepackage{expex}
\usepackage{graphicx}
\usepackage{color}

\raggedbottom
\frenchspacing

\begin{document}           
\title{Documentation for MarginOutline.sty ver. 1.4.1}
\author{Byron Ahn\footnotemark}
\date{February 2019}
\maketitle
\footnotetext{To get my email address, switch the words before and after the period in ahn.byron@gmail.com.} 

\section{Introduction}
\1 This package provides a new means of easily creating indented outlines

\1 Outside of this package, common methods of creating outlines in LaTeX employ itemize or enumerate-type environments.
	\2 In fact, other packages for creating outlines use them too (though perhaps behind the scenes).

\1 Because of this, there are some inherent limitations to what can/cannot and must/must not be done.
	\2 Such outlines are limited in the depth of nesting, and there are potentially undesirable interactions with other packages (e.g. numbered-example packages such as \verb+linguex+ and \verb+expex+)
	\2 Also, such outlines require constant beginning or ending of itemize/enumerate/outline environments
		\3 This makes it clunky to skip around in adding/removing levels of indentation
		\3 It also makes it clunky to employ in \texttt{beamer} presentations

\1 This package defines outlining commands that are not subject to these restrictions.


\section{Package Basics}
\1 There are six options for this package:
	\2[--]{\color{blue}\verb+constantsize+}:	forces constant font size
	\2[--]{\color{blue}\verb+constantexindent+}:	forces all linguex/expex numbered-examples to be uniformly indented
	\2[--]{\color{blue}\verb+allbullets+}:	even outline items at a depth of 1 have bullets (by default, only outline items at a depth of 2 or deeper have bullets)
	\2[--]{\color{blue}\verb+varvertical+}:	items have a variable vertical spacing (when two items of the same depth occur in a row, the spacing is made smaller)
	\2[--]{\color{blue}\verb+beamerpause+}:	when in the \texttt{beamer} class is loaded and this option is invoked, pauses are inserted before each item, \textbf{except} not before the first item on a slide.
	\2[--]{\color{blue}\verb+beamerpauseatstart+}:	when in the \texttt{beamer} class is loaded and this option is invoked, pauses are inserted before each item, \textbf{including} before the first item on a slide.

\0 There are several commands provided as well; the most useful are described below
\1 Outlining basics (section \ref{sec: basics}):
	\2 {\color{blue}\verb+\startoutline+}
		\3 Sets some spacing variables appropriately. (This is \textit{always optional}, and you do not need it to start an outline)

	\2 {\color{blue}\verb+\stopoutline[<length>]+}
		\3 Sets some spacing variables appropriately. (This is \textit{always optional}, and you do not need it to end an outline)
		\3 When a length argument is specified by the optional argument, the indentation for numbered examples \textit{after} the outline is set to that length.
		\3 When unspecified (no optional argument given), the value is \verb+0pt+.

	\2 {\color{blue}\verb+\Outline[<symbol>]{<num>}+}
		\3 For creating outline bullet items, at depth \verb+num+.
		\3 When the optional symbol argument is present, the symbol used for that bullet item is the one given as the argument

	\2 {\color{blue}\verb+\OutlinePause[<symbol>]{<num>}+}
		\3 Same as \verb+\Outline+, except a pause is added between items in \texttt{beamer} documents.
		\3 (More on pausing in section \ref{sec:pausing})

	\2 {\color{blue}\verb+\OutlineMaybePause[<symbol>]{<num>}+}
		\3 Same as \verb+\OutlinePause+, except a pause is added between items if and only if (i) the document is a \texttt{beamer} presentation and (ii) \verb+MarginOutline+ is loaded with the \verb+beamerpause+ option.
		\3 (More on pausing in section \ref{sec:pausing})

	\2 {\color{blue}\verb+\0[<symbol>]+}, {\color{blue}\verb+\1[<symbol>]+}, {\color{blue}\verb+\2[<symbol>]+}, {\color{blue}\verb+\3[<symbol>]+}, {\color{blue}\verb+\4[<symbol>]+},\newline {\color{blue}\verb+\5[<symbol>]+}, {\color{blue}\verb+\6[<symbol>]+}, {\color{blue}\verb+\7[<symbol>]+}, {\color{blue}\verb+\8[<symbol>]+}, {\color{blue}\verb+\9[<symbol>]+}
		\3 \verb+\1+ is equivalent to \verb+\OutlineMaybePause{1}+, \verb+\2+ is equivalent to \verb+\OutlineMaybePause{2}+, etc.
		\3 These are shortcut commands, which \verb+MarginOutline+ attempts to define
			\4 However, if any of them is already defined when \verb+MarginOutline+ is loaded, \verb+MarginOutline+ will not override them.


\1 Adjusting pausing behavior in \texttt{beamer} documents (section \ref{sec:pausing})
	\2 {\color{blue}\verb+\turnOnPauses+} and {\color{blue}\verb+\turnOffPauses+}
		\3 These commands have local scope and affect whether commands like \verb+\OutlineMaybePause+ or \verb+\1+ introduce pauses before them
		\3 ({\itshape{Also provided as equivalents: \color{blue}\verb+\turnPausesOn+} and {\color{blue}\verb+\turnPausesOff+}})

\1 Bullet symbols (section \ref{sec: bullets}):
		
	\2 {\color{blue}\verb+\thefirstbullet+}, {\color{blue}\verb+\thesecondbullet+}, {\color{blue}\verb+\thethirdbullet+}, {\color{blue}\verb+\thefourthbullet+}, \linebreak{\color{blue}\verb+\thefifthbullet+}, {\color{blue}\verb+\thesixthbullet+}, {\color{blue}\verb+\theseventhbullet+}, {\color{blue}\verb+\theeighthbullet+}
		\3 These commands define the symbols used for bulleting

	\2 {\color{blue}\verb+\drarrow+}, {\color{blue}\verb+\pointright+}, {\color{blue}\verb+\cnbullet{<num>}+}, {\color{blue}\verb+\cnbullet*{<num>}+}
		\3 These commands provide shortcuts to handy symbols

\1 Controlling Spacing (section \ref{sec: spacing}):

	\2 {\color{blue}\verb+\bulletspacingfactor{<number>}+}
		\3 Changes the vertical spacing between bullet points. Give it a number value, e.g. \verb+1.5+, to multiply the spacing between lines by 1.5.

	\2 {\color{blue}\verb+\bulletindentfactor{<number>}+}
		\3 Changes the horizontal spacing of bullet point indentation. Give it a number value, e.g. \verb+1.5+, to multiply all the indentations by 1.5.


	\2 {\color{blue}\verb+\exgotodepth{<number>}+}
		\3 When used before an example (either linguex \verb+\ex.+ or expex \verb+\ex \xe+), the example's is indented to the same level of indentation as as the number provided.
		
	\2 {\color{blue}\verb+\exaddedindent{<length>}+}
		\3 Changes how much numbered examples are indented from the (relevant) left margin. Give it a length value, e.g. \verb+12pt+, to make an indent of 12pt.

\1 Finally, the package defines two new environments
	\2 {\color{blue}\verb+\begin{CenterBox}(*)[<color>][<box width>][<outline thickness>]{<header>}+\\\verb+<body>+\\\verb+\end{CenterBox}+}
		\3 Creates a centered, framed box that is centered in the middle of your outline. (See section \ref{sec: centerbox}.)
	

\section{Outline Basics}\label{sec: basics}
\1 Unlike environments, where \verb+\begin{}+/\verb+\end{}+ commands are required, there is no requirement here.
	\2 This structuring was motivated for making the creation of \texttt{beamer} slides easier
		\3 In other outline packages, \verb+\begin+/\verb+\end+ commands are required in every frame that has an outline; this can be cumbersome. 
	\2 The \verb+\startoutline+/\verb+\stopoutline+ commands may be used to begin/end the outline
		\3 Their use is recommended, to ensure proper spacing adjustments take place.
		\3 But using them is not required (and so forgetting them will not crash \LaTeX).
		\3 And there is no problem of requiring a \verb+\stopoutline+ that is local to each \verb+\startoutline+ (unlike \verb+\begin+/\verb+\end+ commands).

\1 Creating outline items is simple; \verb+\Outline{1}+ (shortcut: \verb+\1+) creates an outline item that is aligned with the left margin of the document's text area.
	\2 \verb+\Outline{2}+ (\verb+\2+) creates an outline item that is indented one level to the right, \verb+\Outline{3}+ (\verb+\3+) indents two levels to the right, etc. etc.
	\2 There is no theoretical limit on the depth of nesting
	\2 But beyond \verb+\Outline{9}+, bullets are undefined
		\3 Thus \verb+\Outline{10}+ produces no bullet
			\4 Call it with a defined bullet (e.g. \verb+\Outline[--]{10}+) or define the bullet (see section \ref{sec: bullets})
	\2 (However, as for the shortcut commands, \LaTeX{} does not allow for commands with a numeral in any position but the first. E.g., it would treat \verb+\10+ as \verb+\1+ followed by \verb+0+)

\1 Skipping around in indentation can be done mindlessly
	\2 You can have a \verb+\1+ immediately following a \verb+\8+, or a \verb+\4+ immediately following a \verb+\2+, and the indentation and bulleting is done as you would expect

\1 New \verb+section+s, \verb+subsection+s and \verb+subsubsection+s can be introduced in the middle of an outline
	\2 The section header will be sized according to the document's default (or according to whatever other package is loaded and relevant)
	\2 The header will also be aligned with the page's left margin.  (This is achieved through the \verb+titlesec+ package.)

\newpage

\section{Bulleting Symbols}\label{sec: bullets}
\1 The symbols used for bulleting can be redefined easily.
	\2 Currently, the bullet commands are defined as the following:\\
		\verb+\newcommand{\thefirstbullet}{\ensuremath{\thisblackdiamond}}+\\
		\verb+\newcommand{\thesecondbullet}{\ensuremath{\thissmblacktriangleright}}+\\
		\verb+\newcommand{\thethirdbullet}{\ensuremath{\thissqbullet}}+\\
		\verb+\newcommand{\thefourthbullet}{\ensuremath{\diamond}}+\\
		\verb+\newcommand{\thefifthbullet}{\ensuremath{\thissmalltriangleright}}+\\
		\verb+\newcommand{\thesixthbullet}{\ensuremath{\thissmsquare}}+\\
		\verb+\newcommand{\theseventhbullet}{\ensuremath{\thisvarstar}}+\\
		\verb+\newcommand{\theeighthbullet}{\ensuremath{\circ}}+\\
		\verb+\newcommand{\theninthbullet}{\ensuremath{\bullet}}+
	\2 You can redefine these commands using \verb+\renewcommand+, in the preamble, as normal.

\1 Defining new bullets for deeper levels uses similar logic
	\2 You'll need to define a symbol and edit the \verb+\@bulletinserter+ command.
	\2 The logic of this ought to be transparent.

\1 As with \verb+\item+ in itemize environments, you can define a different symbol for the bullet by specifying an optional argument in []s after the outline level command; e.g. \verb+\Outline[**]{3}+ or \verb+\3[**]+.
	\2 For convenience, there are some additional symbols provided by this package for this kind of custom bulleting:
		\3[\drarrow] \verb+mathabx+'s \verb+\drsh+ (\ensuremath{\thisdrsh}) is provided as \verb+\drarrow+
		\3[\pointright] \verb+pifont+'s \verb+\ding{43}+ (\pointright) is provided as \verb+\pointright+
		\3[\cnbullet{3}] \verb+pifont+'s circle numbers 1-10 (\circleNum{1}-\circleNum{10}) are provided as \verb+\cnbullet{<1-10>}+, and its filled circle numbers 1-10 (\circleNum*{1}-\circleNum*{10}) are provided as \verb+\cnbullet*{<1-10>}+


\section{Controlling Spacing}\label{sec: spacing}
\1 You can control the spacing between items, using \verb+\bulletspacingfactor+.	
	\2 For example, if you would like items to be twice as far apart, invoke the command \verb+\bulletspacingfactor{2}+.
	\2 This command will only make all items that follow it twice as far apart.
	\2 It will also only affect local items; i.e. in a context like:\\
	\phantom{xxx}\verb+\1 { \2 \bulletspacingfactor{2} \3 \3 } \2+\\
	\verb+\bulletspacingfactor{2}+ will \textit{only} be affect the spacing between the first \verb+\2+ and the first \verb+\3+, and between the first \verb+\3+ and the second \verb+\3+ (because of the \verb+{}+s)

\1 Similarly, you can control the spacing between the left margin and the bullet, using \verb+\bulletindentfactor+
	\2 Just the same, invoking the command \verb+\bulletindentfactor{2}+ doubles the spacing
	\2 Like \verb+\bulletspacingfactor+, this only affects items following it locally


\1 Finally, this package is made to be compatible with either of the following two example-numbering packages: \verb+linguex+ and \verb+expex+.
	\2 By default, numbered examples are indented to be left-aligned with the text of the outline bullet in which it occurs.
		\3 e.g., an example after \verb+\Outline{2} <text>+ will be left-aligned with \verb+<text>+
	\2 However, for some it may be more aesthetically pleasing if the numbered examples were left-aligned with the page's left margin.
		\3 To achieve this, load the package with \verb+constantexindent+ option.
	\2 You may want to set the numbered examples to be more/less indented than the default alignment (whether you use \verb+constantexindent+ or not)
		\3 At a \textit{global} level, you can change the indentation with \verb+\exaddedindent{<length>}+.
			\4 If you loaded the \verb+constantexindent+ option, the length you specify will indent examples from the page's left margin.
			\4 If not, the length you specify will indent examples from the text of the outline bullet containing it.
		\3 At a \textit{local} level, you can change the indentation with  \verb+\exgotodepth{<number>}+.
			\4 Use this before the example whose indentation you want to change
			\4 e.g., {\verb+\exgotodepth{2}+} will make it so that the example is left-aligned with the text of \verb+\Outline{2}+ (whether you use \verb+constantexindent+ or not)
			\4 (If \verb+\exaddedindent{<length>}+ has been used previously, the \verb+<length>+ will still be added)

\section{Pausing with \texttt{beamer}}\label{sec:pausing}
\1 Bullet items introduced with \verb+\OutlinePause+ will have a pause inserted before them
	\2 ({\itshape An exception is when \verb+\turnOffPauses+ is used; see below})
\1 \verb+\OutlineMaybePause+ and its derivatives (\verb+\0+--\verb+\9+) can \textbf{\textit{only}} yield a pause in beamer slides \textbf{\textit{if}} the package is loaded with the \verb+beamerpause+ or \verb+beamerpauseatstart+ option 
	\2 These options provide for document-level control over whether these commands introduce pauses before the bullet item
	\2 When \verb+beamerpauseatstart+ is invoked, all bullet points introduced by \verb+\OutlineMaybePause+ or \verb+\0+--\verb+\9+ will be preceded by a pause
		\3 (i.e., Slides appear without the first such bullet point visible)
	\2 \verb+beamerpause+ is similar, except that a pause is \textbf{\textit{not}} introduced by \verb+\OutlineMaybePause+ or \verb+\0+--\verb+\9+ if it is the first bullet point on a slide
		\3 (i.e., Slides appear with the first such bullet point visible)

\1 In addition to these package options, \verb+MarginOutline+ also provides commands to manipulate, mid-document, whether pauses are inserted for outline commands: \verb+\turnOnPauses+ and \verb+\turnOffPauses+
	\2 These two commands affect whether pauses are inserted for the following commands: \verb+\OutlinePause+, \verb+\OutlineMaybePause+, \verb+\0+--\verb+\9+
	\2 \verb+\turnOnPauses+ and \verb+\turnOffPauses+ have an impact on outline commands that come after it
		\3 They have no effect on outline commands that precede them
	\2 Moreover, these commands are locally bounded, meaning they do not extend beyond environments that contain them
		\3 Consider the following beamer document, which produces 19 frames, as the option \verb+beamerpauseatstart+ is invoked and causes a pause before each bullet command:
		\begin{center}
		\fbox{
		\begin{minipage}{.7\linewidth}\footnotesize
			\VerbatimInput{beamerslideexample-2.tex}	
		\end{minipage}
		}
		\end{center}
		\3 If \verb+\turnOffPauses+ were inserted between lines 11 and 12, Bullets 4-8 would all load at once
			\4 (The pauses inced by Bullets 5-8 would be suppressed)
			\4 \textbf{Note}: Because \verb+\turnOffPauses+ is locally bounded, its effects do \textit{not} extend into Slide 2 (its effects are limited to the \verb+frame+ environment that contains it)
		\3 On the other hand, if \verb+\turnOffPauses+ were inserted on line 17, Slides 2 and 3 would not contain any pauses
			\4 This example of \verb+\turnOffPauses+ is not bounded by any local environment, and therefore its effects are exerted over the remainder of the document
		\3 \verb+\turnOffPauses+ and \verb+\turnOnPauses+ interact predictably
			\4 For example, if \verb+\turnOffPauses+ were inserted on line 17 and \verb+\turnOffPauses+ were on line 23, only Slide 2 would be without pauses
			\4 Or, if \verb+\turnOffPauses+ were inserted on line 17 and \verb+\turnOffPauses+ were between lines 26 and 27, outline commands would not insert pauses for Bullets 8--13, but would for bullets 14--16

\section{A Note on Beamer and Section Slides}\label{sec:Slide}
\1 If you use change what happens at the beginning of sections in \verb+beamer+, beware of possible interactions \verb+MarginOutline+
	\2 \verb+MarginOutline+ executes the following:
		\begin{center}
		\begin{minipage}{.7\linewidth}
		\begin{verbatim}
\AtBeginSection{\ZeroOutLeft}
\AtBeginSubsection{\ZeroOutLeft}
\AtBeginSubsubsection{\ZeroOutLeft}	
		\end{verbatim}
		\end{minipage}
		\end{center}
	\2 These \verb+\ZeroOutLeft+ commands ensure proper formatting
		\3 If you make adjustments to what happens at the beginning of sections, you are encouraged to also include \verb+\ZeroOutLeft+

%\1 Occasionally, \verb+MarginOutline+ interacts with other \verb+beamer+ commands in unexpected ways
%	\2 (In particular, title slides and section slides)
%	\2 One way to deal with this is to insert \verb+\startoutline+ command at the beginning of each slide, and a \verb+\stopoutline+ and the end of each slide
%\1 \verb+\MarginOutline+ provides a new environment for slides, which does this for you
%	\2 The \verb+Slide+ environment works like a \verb+frame+ environments, in terms of its arguments
%	\2 That is, the following two environments are largely equivalent, except the latter adds the \verb+MarginOutline+-specific commands for preventing certain formatting-related issues
%		\3[]\begin{minipage}{.33\linewidth}
%	\begin{verbatim}
%\begin{frame}[b]{Test}
%\1 This is a test
%\end{frame}
%	\end{verbatim}	
%\end{minipage}
%~~~~\begin{minipage}{.33\linewidth}
%	\begin{verbatim}
%\begin{Slide}[b]{Test}
%\1 This is a test
%\end{Slide}
%	\end{verbatim}	
%\end{minipage}
			
\section{An Extra Presentation Tool}\label{sec: centerbox}
\1 With the \verb+CenterBox+ environment, you can make boxes that stand out from the rest of your outline
	\2 This is environment takes two optional arguments, and one mandatory argument
\1 To begin a \verb+CenterBox+ environment, use the command:\\\verb+\begin{CenterBox}[<color>][<box width>][<outline thickness>]{<header>}+
	\2 The optional \texttt{color} argument specifies a background color. The default is white.
		\3 (In the case of beamer slides, the default is the background color of your slide.)
	\2 The optional \texttt{box width} argument it specifies the width of the box. The default value is 75\% of the line's width.
		\3 (This argument has no effect in beamer, as beamer \verb+\CenterBox+s are formatted according to the beamertheme.)
	\2 The optional \texttt{outline thickness} argument it specifies the width of the border around the box. The default value is 1pt.
		\3 (This argument has no effect in beamer, as beamer \verb+\CenterBox+s are formatted according to the beamertheme.)
	\2 The \texttt{header} argument puts a header at the top of the \verb+\CenterBox+.
\1 No matter how deeply embedded you are into your outline:
	\2 The CenterBox always appears in the center of the page
		\3 This is exemplified below:
		\begin{CenterBox}{CenterBox Demo}
			CenterBox works like this			
		\end{CenterBox}
		
			\4 This was produced with the code:
				\5[]\begin{minipage}{.5\linewidth}
			\begin{verbatim}
\begin{CenterBox}{CenterBox Demo}
CenterBox works like this			
\end{CenterBox}
			\end{verbatim}	
\end{minipage}

	\2 The \verb+CenterBox+ below was created using the optional arguments
\begin{CenterBox}[red!10][.65\linewidth][2pt]{CenterBox Demo}
This CenterBox has made use of optional arguments
\end{CenterBox}
			\4 This was produced with the code:
				\5[]\begin{minipage}{.5\linewidth}
			\begin{verbatim}
\begin{CenterBox}[red!10][.65\linewidth][2pt]{CenterBox Demo}
This CenterBox has made use of optional arguments
\end{CenterBox}
			\end{verbatim}	
\end{minipage}


	\2 And here is an example of \verb+\CenterBox+ in action on a Beamer slide, produced with the same code for the \verb+CenterBox+ environment
		\begin{center}
			\fboxsep0pt\fbox{\includegraphics[width=.5\linewidth]{beamerslideexample.pdf}}
		\end{center}
		
		\3 The full .tex for this beamer document above:
		\begin{center}
		\fbox{
		\begin{minipage}{.7\linewidth}\footnotesize
			\VerbatimInput{beamerslideexample.tex}	
		\end{minipage}
		}
		\end{center}

\1 In addition, if you add a * after \verb+\begin{CenterBox}+, the header will be suppressed:
	\2 Here is an example:
		\begin{CenterBox}*[yellow][.65\linewidth][.5pt]{CenterBox Demo}
			A CenterBox with no header works like this			
		\end{CenterBox}
			\2 This was produced with the code:\\\begin{minipage}{.5\linewidth}
			\begin{verbatim}
\begin{CenterBox}*[yellow][.65\linewidth][.5pt]{CenterBox Demo}
CenterBox works like this			
\end{CenterBox}
			\end{verbatim}
			\end{minipage}
		\3 \textbf{NOTE}: Even though the header is suppressed in the output, you \textit{must} include the header argument (if only as curly braces without any content: i.e. \{\})
			\4 If you do not, you may get some errors

\1 Lastly, similar to \verb+CenterBox+ environments, there is the nearly identical \verb+BigBox+ environment
	\2 It makes use of all the same optional and mandatory arguments
	\2 The only difference is that, by default, the width of the box is almost an entire line's width (90\%)


\end{document}