% MarginOutline documentation
% created by Byron Ahn
% email address: switch the words before and after the period in ahn.byron@gmail.com
% 2014.04.07
\documentclass[11pt]{article}
\usepackage[varvertical,constantsize,allbullets]{MarginOutline}
\usepackage[margin=1.25in]{geometry}
\usepackage[bottom]{footmisc}
\usepackage{expex}
\usepackage{graphicx}
\usepackage{color}

\raggedbottom
\frenchspacing

\begin{document}           
\title{Documentation for MarginOutline.sty ver. 1.3}
\author{Byron Ahn\footnotemark}
\date{October 2017}
\maketitle
\footnotetext{To get my email address, switch the words before and after the period in ahn.byron@gmail.com.} 

\section{Introduction}
\1 This package provides a new means of easily creating indented outlines

\1 Outside of this package, common methods of creating outlines in LaTeX employ itemize or enumerate-type environments.
	\2 In fact, other packages for creating outlines use them too (though perhaps behind the scenes).

\1 Because of this, there are some inherent limitations to what can/cannot and must/must not be done.
	\2 Such outlines are limited in the depth of nesting, and there are potentially undesirable interactions with other packages (e.g. numbered-example packages such as \verb+linguex+ and \verb+expex+)
	\2 Also, such outlines require constant beginning or ending of itemize/enumerate/outline environments
		\3 This makes it clunky to skip around in adding/removing levels of indentation
		\3 It also makes it clunky to employ in beamer presentations

\1 This package defines outlining commands that are not subject to these restrictions.


\section{Package Basics}
\1 There are six options for this package:
	\2[--]{\color{blue}\verb+constantsize+}:	forces constant font size
	\2[--]{\color{blue}\verb+constantexindent+}:	forces all linguex/expex numbered-examples to be uniformly indented
	\2[--]{\color{blue}\verb+allbullets+}:	even outline items at a depth of 1 have bullets (by default, only outline items at a depth of 2 or deeper have bullets)
	\2[--]{\color{blue}\verb+varvertical+}:	items have a variable vertical spacing (when two items of the same depth occur in a row, the spacing is made smaller)
	\2[--]{\color{blue}\verb+beamerpause+}:	when in the beamer class is loaded and this option is invoked, pauses are inserted between items.
	\2[--]{\color{blue}\verb+beamerpauseatstart+}:	when in the beamer class is loaded and this option is invoked, pauses are inserted before (and not just in between) all items.

\1 There are several commands provided as well; the most useful are described below
\1 Outlining basics (section \ref{sec: basics}):
	\2 {\color{blue}\verb+\startoutline+}
		\3 Sets some spacing variables appropriately. (This is \textit{always optional}, and you do not need it to start an outline)

	\2 {\color{blue}\verb+\stopoutline[<length>]+}
		\3 Sets some spacing variables appropriately. (This is \textit{always optional}, and you do not need it to end an outline)
		\3 When a length argument is specified by the optional argument, the indentation for numbered examples \textit{after} the outline is set to that length.
		\3 When unspecified (no optional argument given), the value is \verb+0pt+.

	\2 {\color{blue}\verb+\Outline[<symbol>]{<num>}+}
		\3 For creating outline bullet items, at depth \verb+num+.
		\3 When the optional symbol argument is present, the symbol used for that bullet item is the one given as the argument

	\2 {\color{blue}\verb+\OutlinePause[<symbol>]{<num>}+}
		\3 Same as \verb+\Outline+, except a pause is added between items in beamer presentations.

	\2 {\color{blue}\verb+\OutlineMaybePause[<symbol>]{<num>}+}
		\3 Same as \verb+\OutlinePause+, except a pause is added between items if and only if (i) the document is a beamer presentation and (ii) \verb+MarginOutline+ is loaded with the \verb+beamerpause+ option.

	\2 {\color{blue}\verb+\0[<symbol>]+}, {\color{blue}\verb+\1[<symbol>]+}, {\color{blue}\verb+\2[<symbol>]+}, {\color{blue}\verb+\3[<symbol>]+}, {\color{blue}\verb+\4[<symbol>]+},\newline {\color{blue}\verb+\5[<symbol>]+}, {\color{blue}\verb+\6[<symbol>]+}, {\color{blue}\verb+\7[<symbol>]+}, {\color{blue}\verb+\8[<symbol>]+}, {\color{blue}\verb+\9[<symbol>]+}
		\3 These are shortcut commands, which \verb+MarginOutline+ attempts to define
			\4 However, if any of them is already defined when \verb+MarginOutline+ is loaded, \verb+MarginOutline+ will not override them.
		\3 \verb+\1+ is equivalent to \verb+\OutlineMaybePause{1}+, \verb+\2+ is equivalent to \verb+\OutlineMaybePause{2}+, etc.

\1 Bullet symbols (section \ref{sec: bullets}):
		
	\2 {\color{blue}\verb+\thefirstbullet+}, {\color{blue}\verb+\thesecondbullet+}, {\color{blue}\verb+\thethirdbullet+}, {\color{blue}\verb+\thefourthbullet+}, \linebreak{\color{blue}\verb+\thefifthbullet+}, {\color{blue}\verb+\thesixthbullet+}, {\color{blue}\verb+\theseventhbullet+}, {\color{blue}\verb+\theeighthbullet+}
		\3 These commands define the symbols used for bulleting

	\2 {\color{blue}\verb+\drarrow+}, {\color{blue}\verb+\pointright+}, {\color{blue}\verb+\cnbullet{<num>}+}, {\color{blue}\verb+\cnbullet*{<num>}+}
		\3 These commands provide shortcuts to handy symbols

\1 Controlling Spacing (section \ref{sec: spacing}):

	\2 {\color{blue}\verb+\bulletspacingfactor{<number>}+}
		\3 Changes the vertical spacing between bullet points. Give it a number value, e.g. \verb+1.5+, to multiply the spacing between lines by 1.5.

	\2 {\color{blue}\verb+\bulletindentfactor{<number>}+}
		\3 Changes the horizontal spacing of bullet point indentation. Give it a number value, e.g. \verb+1.5+, to multiply all the indentations by 1.5.


	\2 {\color{blue}\verb+\exgotodepth{<number>}+}
		\3 When used before an example (either linguex \verb+\ex.+ or expex \verb+\ex \xe+), the example's is indented to the same level of indentation as as the number provided.
		
	\2 {\color{blue}\verb+\exaddedindent{<length>}+}
		\3 Changes how much numbered examples are indented from the (relevant) left margin. Give it a length value, e.g. \verb+12pt+, to make an indent of 12pt.

%	\2 {\color{blue}\verb+\toofull[<length>]+}
%		\3 When a length argument is specified by the optional argument, the amount of (negative) space can be controlled. (Otherwise, the value is \verb+-10em+).

\1 Finally, the package defines two new environments
	\2 {\color{blue}\verb+\begin{CenterBox}[<color>][<width>]{<header>}+\\\verb+<body>+\\\verb+\end{CenterBox}+}
		\3 Creates a centered, framed box that is centered in the middle of your outline. (See section \ref{sec: centerbox}.)

	\2 {\color{blue}\verb+\begin{frameoutline}[<optional paramters>]+\\\verb+\end{frameoutline}+}
		\3 Mimics a beamer frame, with some added formatting tools to make these outline commands work better. (See section \ref{sec:frameoutline}.) The optional parameters are the same as those provided to the \verb+frame+ environment, as defined by beamer.
		\3 NOTE: \textit{This only works when used in a beamer document}
	

\section{Outline Basics}\label{sec: basics}
\1 Unlike environments, where \verb+\begin{}+/\verb+\end{}+ commands are required, there is no requirement here.
	\2 This structuring was motivated for making the creation of beamer slides easier
		\3 In other outline packages, \verb+\begin+/\verb+\end+ commands are required in every frame that has an outline; this can be cumbersome. 
	\2 The \verb+\startoutline+/\verb+\stopoutline+ commands may be used to begin/end the outline
		\3 Their use is recommended, to ensure proper spacing adjustments take place.
		\3 But using them is not required (and so forgetting them will not crash \LaTeX).
		\3 And there is no problem of requiring a \verb+\stopoutline+ that is local to each \verb+\startoutline+ (unlike \verb+\begin+/\verb+\end+ commands).

\1 Creating outline items is simple; \verb+\Outline{1}+ (shortcut: \verb+\1+) creates an outline item that is aligned with the left margin of the document's text area.
	\2 \verb+\Outline{2}+ (\verb+\2+) creates an outline item that is indented one level to the right, \verb+\Outline{3}+ (\verb+\3+) indents two levels to the right, etc. etc.
	\2 There is no theoretical limit on the depth of nesting
	\2 But beyond \verb+\Outline{9}+, bullets are undefined
		\3 Thus \verb+\Outline{10}+ produces no bullet
			\4 Call it with a defined bullet (e.g. \verb+\Outline[--]{10}+) or define the bullet (see section \ref{sec: bullets})
	\2 (However, as for the shortcut commands, \LaTeX{} does not allow for commands with a numeral in any position but the first. E.g., it would treat \verb+\10+ as \verb+\1+ followed by \verb+0+)

\1 Skipping around in indentation can be done mindlessly
	\2 You can have a \verb+\1+ immediately following a \verb+\8+, or a \verb+\4+ immediately following a \verb+\2+, and the indentation and bulleting is done as you would expect

\1 New \verb+section+s, \verb+subsection+s and \verb+subsubsection+s can be introduced in the middle of an outline
	\2 The section header will be sized according to the document's default (or according to whatever other package is loaded and relevant)
	\2 The header will also be aligned with the page's left margin.  (This is achieved through the \verb+titlesec+ package.)

\newpage
\section{Bulleting Symbols}\label{sec: bullets}
\1 The symbols used for bulleting can be redefined easily.
	\2 Currently, the bullet commands are defined as the following:\\
		\verb+\newcommand{\thefirstbullet}{\ensuremath{\thisblackdiamond}}+\\
		\verb+\newcommand{\thesecondbullet}{\ensuremath{\thissmblacktriangleright}}+\\
		\verb+\newcommand{\thethirdbullet}{\ensuremath{\thissqbullet}}+\\
		\verb+\newcommand{\thefourthbullet}{\ensuremath{\diamond}}+\\
		\verb+\newcommand{\thefifthbullet}{\ensuremath{\thissmalltriangleright}}+\\
		\verb+\newcommand{\thesixthbullet}{\ensuremath{\thissmsquare}}+\\
		\verb+\newcommand{\theseventhbullet}{\ensuremath{\thisvarstar}}+\\
		\verb+\newcommand{\theeighthbullet}{\ensuremath{\circ}}+\\
		\verb+\newcommand{\theninthbullet}{\ensuremath{\bullet}}+
	\2 You can redefine these commands using \verb+\renewcommand+, in the preamble, as normal.

\1 Defining new bullets for deeper levels uses similar logic
	\2 You'll need to define a symbol and edit the \verb+\@bulletinserter+ command.
	\2 The logic of this ought to be transparent.

\1 As with \verb+\item+ in itemize environments, you can define a different symbol for the bullet by specifying an optional argument in []s after the outline level command; e.g. \verb+\Outline[**]{3}+ or \verb+\3[**]+.
	\2 For convenience, there are some additional symbols provided by this package for this kind of custom bulleting:
		\3[\drarrow] \verb+mathabx+'s \verb+\drsh+ (\ensuremath{\thisdrsh}) is provided as \verb+\drarrow+
		\3[\pointright] \verb+pifont+'s \verb+\ding{43}+ (\pointright) is provided as \verb+\pointright+
		\3[\cnbullet{3}] \verb+pifont+'s circle numbers 1-10 (\circleNum{1}-\circleNum{10}) are provided as \verb+\cnbullet{<1-10>}+, and its filled circle numbers 1-10 (\circleNum*{1}-\circleNum*{10}) are provided as \verb+\cnbullet*{<1-10>}+


\section{Controlling Spacing}\label{sec: spacing}
\1 You can control the spacing between items, using \verb+\bulletspacingfactor+.	
	\2 For example, if you would like items to be twice as far apart, invoke the command \verb+\bulletspacingfactor{2}+.
	\2 This command will only make all items that follow it twice as far apart.
	\2 It will also only affect local items; i.e. in a context like:\\
	\phantom{xxx}\verb+\1 { \2 \bulletspacingfactor{2} \3 \3 } \2+\\
	\verb+\bulletspacingfactor{2}+ will \textit{only} be affect the spacing between the first \verb+\2+ and the first \verb+\3+, and between the first \verb+\3+ and the second \verb+\3+ (because of the \verb+{}+s)

\1 Similarly, you can control the spacing between the left margin and the bullet, using \verb+\bulletindentfactor+
	\2 Just the same, invoking the command \verb+\bulletindentfactor{2}+ doubles the spacing
	\2 Like \verb+\bulletspacingfactor+, this only affects items following it locally


\1 Finally, this package is made to be compatible with either of the following two example-numbering packages: \verb+linguex+ and \verb+expex+.
%	\2 In other outlining methods that rely upon itemize/enumerate....
%		\3 ...\verb+linguex+ examples indent to the level of indentation of the preceding itemize/enumerate item, and
%		\3 ...\verb+expex+ examples do not indent in the same way.
	\2 By default, numbered examples are indented to be left-aligned with the text of the outline bullet in which it occurs.
		\3 e.g., an example after \verb+\Outline{2} <text>+ will be left-aligned with \verb+<text>+
	\2 However, for some it may be more aesthetically pleasing if the numbered examples were left-aligned with the page's left margin.
		\3 To achieve this, load the package with \verb+constantexindent+ option.
	\2 You may want to set the numbered examples to be more/less indented than the default alignment (whether you use \verb+constantexindent+ or not)
		\3 At a \textit{global} level, you can change the indentation with \verb+\exaddedindent{<length>}+.
			\4 If you loaded the \verb+constantexindent+ option, the length you specify will indent examples from the page's left margin.
			\4 If not, the length you specify will indent examples from the text of the outline bullet containing it.
		\3 At a \textit{local} level, you can change the indentation with  \verb+\exgotodepth{<number>}+.
			\4 Use this before the example whose indentation you want to change
			\4 e.g., {\verb+\exgotodepth{2}+} will make it so that the example is left-aligned with the text of \verb+\Outline{2}+ (whether you use \verb+constantexindent+ or not)
			\4 (If \verb+\exaddedindent{<length>}+ has been used previously, the \verb+<length>+ will still be added)

\section{A Note for Use with Beamer}\label{sec:frameoutline}
\1 \verb+\OutlineMaybePause+ and its derivatives (\verb+\1+, \verb+\2+, \verb+\3+, ...) rely on counters that check what the last outline command was
	\2 If \verb+MarginOutline+ detects that this is the first outline command...
		\3 ...and the \verb+beamerpause+ option is loaded, but \verb+beamerpauseatstart+ is not, no pause will be inserted
		\3 ...and the \verb+beamerpauseatstart+ option is loaded, a pause will be inserted
\1 You may want \verb+MarginOutline+ to treat some outline commands as `the first outline command' for purposes of pause-insertion
	\2 For example, you may want the first outline command on each slide to be treated this way
	\2 To do this, you can insert a \verb+\ZeroOutLeft+ command at the beginning of each slide
		\3 Alternatively, you can define your own environment for slides, which executes \verb+\ZeroOutLeft+ every time
		\3 Such an environment -- \verb+frameoutline+ -- is provided by this package

			
\section{An Extra Presentation Tool}\label{sec: centerbox}
%\1 Some extra commands are provided to help make good handouts/presentations
\1 With the \verb+CenterBox+ environment, you can make boxes that stand out from the rest of your outline
	\2 This is environment takes two optional arguments, and one mandatory argument
\1 To begin a \verb+CenterBox+ environment, use the command:\\\verb+\begin{CenterBox}[<color>][<width>]{<header>}+
	\2 The optional \texttt{color} argument specifies a background color. The default is white.
		\3 (In the case of beamer slides, the default is the background color of your slide.)
	\2 The optional \texttt{width} argument it specifies the width of the border around the box. The default value is 1pt.
		\3 (This argument has no effect in beamer, as beamer \verb+\CenterBox+s are formatted according to the beamertheme.)
	\2 The \texttt{header} argument puts a header at the top of the \verb+\CenterBox+.
\1 No matter how deeply embedded you are into your outline:
	\2 The CenterBox always appears in the center of the page, at a width of 75\% of the line width
		\3 This is exemplified below:
		\begin{CenterBox}[green!25][3pt]{CenterBox Demo}
			CenterBox works like this			
		\end{CenterBox}
		
			\4 This was produced with the code:\\\begin{minipage}{.5\linewidth}
			\begin{verbatim}
\begin{CenterBox}[green!25][3pt]{CenterBox Demo}
CenterBox works like this			
\end{CenterBox}
			\end{verbatim}	
\end{minipage}


	\2 And here is an example of \verb+\CenterBox+ in action on a Beamer slide, produced with the same code for the \verb+CenterBox+ environment
		\begin{center}
			\fboxsep0pt\fbox{\includegraphics[width=.5\linewidth]{beamerslideexample.pdf}}
		\end{center}
		
		\3 The full .tex for this beamer document above:
		\begin{center}
		\begin{minipage}{.7\linewidth}\footnotesize
			\begin{verbatim}
\documentclass[handout]{beamer}
\usetheme{Madrid}
\usepackage[allbullets]{MarginOutline}

\begin{document}           

\begin{frame}{A Beamer Example}
\1 This is a beamer slide
	\2 It has a CenterBox on it
\begin{CenterBox}[green!25][3pt]{CenterBox Demo}
CenterBox works like this			
\end{CenterBox}
\1 The exact formatting depends on how the beamer theme you choose
formats the block environment
\end{frame}
			
\end{document}
			\end{verbatim}
		\end{minipage}
		\end{center}

\1 In addition, if you add a * after \verb+\begin{CenterBox}+, the header will be suppressed:
	\2 Here is an example:
		\begin{CenterBox}*[green!25][3pt]{CenterBox Demo}
			CenterBox works like this			
		\end{CenterBox}
			\2 This was produced with the code:\\\begin{minipage}{.5\linewidth}
			\begin{verbatim}
\begin{CenterBox}*[green!25][3pt]{CenterBox Demo}
CenterBox works like this			
\end{CenterBox}
			\end{verbatim}
			\end{minipage}
		\3 \textbf{NOTE}: Even though the header is suppressed in the output, you \textit{must} include the header argument (if only as curly braces without any content: i.e. \{\})
			\4 If you do not, you may get some errors




\end{document}